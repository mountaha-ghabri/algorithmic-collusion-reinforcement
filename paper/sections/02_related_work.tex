\section{Related Work}

\subsection{Emergence of Algorithmic Collusion}

The foundational work on algorithmic collusion comes from Calvano, Calzolari, Denicolò, and Pastorello (2020), who demonstrated that independent Q-learning agents in repeated Bertrand competition can converge to supra-competitive prices without communication or explicit coordination~\cite{calvano2020artificial}. Their key finding was that simple reinforcement learning algorithms, when used independently by competing firms, can discover and sustain tacitly collusive strategies that yield prices substantially above competitive levels.

Subsequent work has extended and validated these findings. Klein (2021) showed that collusion persists even with asynchronous price updates and sequential decision-making, addressing concerns about the simultaneity assumption in the original model~\cite{klein2021autonomous}. Abada, Lambin, and Tóth (2023) provided additional evidence of algorithmic collusion across different market structures and confirmed that these findings are robust to various parameter specifications~\cite{abada2023artificial}.

Empirical evidence complements these simulation studies. Assad, Clark, Ershov, and Xu (2024) analyzed gasoline pricing data and found correlations between algorithmic pricing adoption and patterns consistent with tacit collusion, though they note the difficulty of establishing causal relationships in observational data~\cite{assad2024algorithmic}.

\subsection{Multi-Agent Reinforcement Learning in Economics}

The broader literature on multi-agent reinforcement learning (MARL) provides theoretical context for these findings. Leibo et al. (2017) showed that independent learners in social dilemma games can develop cooperative strategies through repeated interaction, even when optimizing individual rewards~\cite{leibo2017multi}. This aligns with economic theories of tacit collusion in repeated games, where cooperation can emerge as an equilibrium without explicit coordination.

However, as Dafoe et al. (2020) note, cooperation in MARL is not guaranteed and depends critically on environmental factors, reward structures, and learning algorithms~\cite{dafoe2020open}. The specific conditions under which algorithmic collusion emerges remain an active research area.

\subsection{Algorithmic Pricing and Antitrust Policy}

The policy implications of algorithmic collusion have been extensively discussed. Ezrachi and Stucke (2016) warned about the potential for algorithms to facilitate anticompetitive coordination, coining the term "digital eye" to describe how pricing algorithms might monitor and respond to competitors in ways that sustain collusion~\cite{ezrachi2016virtual}. 

Baker (2021) argues that existing antitrust laws are adequate to address algorithmic collusion but acknowledges enforcement challenges~\cite{baker2021algorithms}. By contrast, Mehra (2016) suggests that algorithmic coordination may require new regulatory approaches, particularly when collusion emerges from independent learning rather than explicit agreement~\cite{mehra2016antitrust}.

\subsection{Gaps in the Literature}

While the emergence of algorithmic collusion is well-documented, its \emph{stability} remains understudied. Calvano et al. (2020) briefly test single-period deviations and find quick reversion to collusive prices, but they do not systematically analyze different intervention types or measure recovery dynamics~\cite{calvano2020artificial}. Klein (2021) examines robustness to parameter variations but not to deliberate interventions~\cite{klein2021autonomous}.

No prior work systematically tests:
\begin{itemize}
    \item Multiple intervention types (forced competitive pricing, exploration shocks, memory resets) on established collusion
    \item Quantitative recovery rates and reinforcement effects
    \item Comparative effectiveness of different regulatory-style interventions
    \item Statistical significance of post-intervention price changes
\end{itemize}

This paper addresses these gaps by providing the first systematic analysis of algorithmic collusion stability under multiple intervention types, with particular attention to whether interventions disrupt or paradoxically reinforce collusive behavior.