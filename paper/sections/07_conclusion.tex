\section{Conclusion}

This paper investigates the stability of algorithmic tacit collusion under regulatory-style interventions. We demonstrate that independent Q-learning agents in repeated Bertrand competition not only converge to supra-competitive pricing (average price = 1.561, collusion index = 0.298) but exhibit troubling stability when subjected to interventions designed to disrupt coordination.

Our key finding is a \emph{collusion reinforcement paradox}: rather than disrupting collusion, interventions consistently strengthen it. Forced competitive pricing increased prices by 3.6–3.7\%, exploration shocks increased prices by 2.2\%, and memory resets increased prices by 3.4\%. All interventions yielded recovery rates exceeding 110\% (range: 111.8–119.6\%, average: 117.2\%), with statistical significance confirmed across all tests ($p < 0.05$).

These results challenge conventional antitrust wisdom and suggest that traditional regulatory tools, fines, audits, and mandatory updates, may be inadequate or even counterproductive for algorithmic markets. Algorithmic collusion appears to exhibit self-reinforcing properties where disruptions lead to more robust coordination, contrasting sharply with human collusion which typically collapses under regulatory pressure.

The policy implications are significant. Regulators must shift from reactive disruption to preventive design, focusing on algorithmic diversity, ex-ante approval processes, and enhanced monitoring capabilities. As pricing algorithms become more prevalent and sophisticated, developing effective regulatory frameworks for algorithmic competition becomes increasingly urgent.

Future research should explore more complex market structures, heterogeneous algorithms, and novel intervention strategies. The alternative, relying on regulatory tools designed for human coordination in markets increasingly dominated by autonomous learning agents, risks creating environments where collusion becomes not just possible but self-reinforcing and resistant to traditional interventions.

Our findings highlight the need for interdisciplinary collaboration between computer scientists, economists, and legal scholars to develop regulatory approaches that account for the unique properties of algorithmic markets. Only through such collaboration can we hope to maintain competitive markets in the age of autonomous pricing agents.